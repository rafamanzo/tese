\documentclass[a4paper,11pt]{report}
\usepackage[T1]{fontenc}
\usepackage[utf8]{inputenc}
\usepackage{lmodern}
\usepackage[english]{babel}
\usepackage{amsfonts}
\usepackage{hyperref}

\title{Work report from 12/12/2013 to 10/01/2014}
\author{Rafael Reggiani Manzo}

\begin{document}

\maketitle
\tableofcontents

\begin{abstract}
\end{abstract}

\chapter{Report}
\section{Previous meeting}
  \subsection{What was presented}

  \subsection{Next steps}

\section{More references}
This was not part of what we've discussed on our last meeting, but I'm still feeling that I haven't read enough yet. So, I've found three new articles that I'd like to discuss:

\begin{enumerate}
  \item Probabilistic Monte Carlo Based Mapping of Cerebral Connections Utilising Whole-Brain Crossing Fibre Information. Geoff J. M. Parker, Daniel C. Alexander
  \item Diffusion Anisotropy Measurement of Brain White Matter Is Affected by Voxel Size: Underestimation Occurs in Areas with Crossing Fibers. H. Oouchi, K. Yamada, K. Sakai, O. Kizu, T. Kubota, H. Ito, T. Nishimura
  \item Fiber Crossing in Human Brain Depicted with Diffusion Tensor MR Imaging. Mette R. Wiegell, Henrik B. W. Larsson, Van J. Wedeen
\end{enumerate}

The number 1 is probably something really similar to what I'm trying to achieve: a pre-mapping of fiber crossing regions to improve tractography's accuracy on that regions. The main difference is the probabilistic approach that it proposes. It's an article from 2003 with lots of spelling errors, so I'm not sure about it's credibility.

The second and third articles were interesting to bring to my mind some aspects of the problem that I haven't thought about before. The article number two shows the FA values, for some brain regions well known as crossing regions, may vary depending on the voxel dimensions. Finally with the number three I've learned that, besides the cases of simple fiber crossing and fiber kissing, there two other cases that I've not considered yet: fiber spreading; and high curvature fibers.

\end{document}