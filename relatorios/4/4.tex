\documentclass[a4paper,11pt]{report}
\usepackage[T1]{fontenc}
\usepackage[utf8]{inputenc}
\usepackage{lmodern}
\usepackage[english]{babel}
\usepackage{amsfonts}
\usepackage{hyperref}
\usepackage{graphicx}
\usepackage{subcaption}
\usepackage{float}

\title{Technical report from 08/08/2014 to 15/08/2014}
\author{Rafael Reggiani Manzo}

\begin{document}

\maketitle
\tableofcontents

\chapter{Previous meeting}
  \section{What was presented}
  Last time when we met, I've showed the results that I've achieved on a mask obtained from a fibre tracking with the stopping criteria at angles higher than 60 degrees instead of 45 degrees and then cropped to a small region of interest.

  The results using clustering algorithms showed no visually meaningful results for this small region. But, on the other hand, a implementation of a segmentation algorithm was capable of segmenting the corner region of the region of interest.

  \section{Next steps}
  From this we agreed that the next steps would be to first produce the maps the three eigenvalues of the tensor ($\lambda_1$, $\lambda_2$, $\lambda_3$), apply the same segmentation algorithm, then apply it to the FA, MD, RD and TV maps.

  After that, produce a prototype that accordingly to difference between the eigenvectors in order to produce a threshold mask. This should be configurable with parameters that may vary from full isotropy to full anisotropy.

\chapter{Work done}
  All the work expected from the last meeting was done in time as expected.

  \section{Eigenvalues differences threshold}
    Every index that we can extract from a tensor is associated to it's eigenvalues or is a combination of them. So looking at them, instead to derived indexes like FA, should provide more information about the dataset.

    Since the eigenvalues will be object of various discussions ahead we shall clarify that we are referencing the three eigenvalues as $\lambda_1$, $\lambda_2$ and $\lambda_3$ in decreasing order.

    \subsection{Procedure}
    In more details, what we looking for when trying to map uncertainty and certainty on a dataset, is the differences between those values. For example:

    \begin{itemize}
      \item $\lambda_1 - \lambda_2 = 1.0$ and $\lambda_1 - \lambda_3 = 1.0$ and $\lambda_2 - \lambda_3 = 0.0$ represent what we call full anisotropy or, as we will reference to it later, perfect anisotropy;
      \item similarly, $\lambda_1 - \lambda_2 = 0.0$ and $\lambda_1 - \lambda_3 = 0.0$ and $\lambda_2 - \lambda_3 = 0.0$ represent what we call full isotropy or, as we will reference to it later, perfect isotropy.
    \end{itemize}

    Those represent the two kinds of region we are pursuing to map: perfect anisotropy is the most certainty region; and perfect isotropy is the most uncertainty region.

    Given this concept, we must consider that both cases never happen in real datasets. So this brings us a relaxation by a parameter $0 \leq \epsilon \leq 1$ for the expressions previously mentioned:

    \newpage
    \begin{itemize}
      \item $1 - \epsilon \leq \lambda_1 - \lambda_2 \leq 1.0$ and $1 - \epsilon \leq \lambda_1 - \lambda_3 \leq 1.0$ and $0.0 \leq \lambda_2 - \lambda_3 \leq \epsilon$;
      \item $0.0 \leq \lambda_1 - \lambda_2 \leq \epsilon$ and $0.0 \leq \lambda_1 - \lambda_3 \leq \epsilon$ and $0.0 \leq \lambda_2 - \lambda_3 \leq \epsilon$.
    \end{itemize}

    Then, by increasing the $\epsilon$ step by step we may find a threshold for approximations of perfect anisotropy and isotropy.

    \subsection{Results}
      \subsubsection{Perfect anisotropy and isotropy}
      \subsubsection{Relaxed perfect anisotropy and isotropy}

  \section{Watersheds}
    \subsection{Fractional anisotropy (FA)}
    \subsection{Mean diffusivity (MD)}
    \subsection{Radial diffusivity (RD)}
    \subsection{Toroidal Volume (TV)}
    \subsection{First eigenvalue ($\lambda_1$)}
    \subsection{Second eigenvalue ($\lambda_2$)}
    \subsection{Third eigenvalue ($\lambda_3$)}

\chapter{Conclusion}

\end{document}