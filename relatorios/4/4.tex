\documentclass[a4paper,11pt]{report}
\usepackage[T1]{fontenc}
\usepackage[utf8]{inputenc}
\usepackage{lmodern}
\usepackage[english]{babel}
\usepackage{amsfonts}
\usepackage{hyperref}
\usepackage{graphicx}
\usepackage{subcaption}
\usepackage{float}

\title{Technical report from 08/08/2014 to 15/08/2014}
\author{Rafael Reggiani Manzo}

\begin{document}

\maketitle
\tableofcontents

\chapter{Previous meeting}
  \section{What was presented}
  Last time when we met, I've showed the results that I've achieved on a mask obtained from a fibre tracking with the stopping criteria at angles higher than 60 degrees instead of 45 degrees and then cropped to a small region of interest.

  The results using clustering algorithms showed no visually meaningful results for this small region. But, on the other hand, a implementation of a segmentation algorithm was capable of segmenting the corner region of the region of interest.

  \section{Next steps}
  From this we agreed that the next steps would be to first produce the maps the three eigenvalues of the tensor ($\lambda_1$, $\lambda_2$, $\lambda_3$), apply the same segmentation algorithm, then apply it to the FA, MD, RD and TV maps.

  After that, produce a prototype that accordingly to difference between the eigenvectors in order to produce a threshold mask. This should be configurable with parameters that may vary from full isotropy to full anisotropy.

\chapter{Work done}
  \section{Eigenvalues differences threshold}
    \subsection{Procedure}
    \subsection{Results}
    \subsubsection{Perfect anisotropy and isotropy}
    \subsubsection{Relaxed perfect anisotropy and isotropy}

  \section{Watersheds}
    \subsection{Fractional anisotropy (FA)}
    \subsection{Mean diffusivity (MD)}
    \subsection{Radial diffusivity (RD)}
    \subsection{Toroidal Volume (TV)}
    \subsection{First eigenvalue ($\lambda_1$)}
    \subsection{Second eigenvalue ($\lambda_2$)}
    \subsection{Third eigenvalue ($\lambda_3$)}

\chapter{Conclusion}

\end{document}