\documentclass[a4paper,11pt]{report}
\usepackage[T1]{fontenc}
\usepackage[utf8]{inputenc}
\usepackage{lmodern}
\usepackage[english]{babel}
\usepackage{amsfonts}
\usepackage{hyperref}

\title{MSc project\\
       \textbf{Front evolution tractography making use of uncertainty probability distributions}}
\author{Rafael Reggiani Manzo\\
        \textbf{Supervisor:} Prof. Doutor Marcel Parolin Jackowski}

\begin{document}

\maketitle
\tableofcontents

\begin{abstract}
Between the many techniques used for medical image exploration, tractography is based on magnetic resonance images in order to reconstruct in a three-dimensional model the fibers of the human body. It is the only non-invasive technique available for human brain studies.

Two of these algorithms are the probabilist tractography and the front evolution tractography, successfully satisfying the objective above. So, is raised the question about how to incorporate the uncertainty probability distributions from the first method on the second one and which kind of results would be possible to achieve.

To answer these questions, an whole theoretical study about the methods must be done, because these methods advanced physics and calculus topics. After studying these methods, will come a stage of implementation, where is expected to be able to, in fact, assimilate both the methodologies.

To finalize, after dominate both techniques, it'll be possible to make the described method combination and to study it's results compared to the other methods.
\end{abstract}

\chapter{Introduction}
  \section{Motivation}
  Tractography is a medical image exploration computational technique that aims, through many different approaches reconstruct the brain white matter fibers (axonal fibers involved on myelin). Being the only non invasive technique for observation \textit{in vivo} of the human brain and one of the most promissory tools to aid in it's study.

  This technique uses as data source magnetic resonance images. In virtue of the white matter nature, where the water molecules diffusivity is more intense in the fiber orientation. This property is called anisotropy.

  On the other hand, the magnetic resonance is subject to noise and it doesn't represents with precision the fiber crossing. So, over time, were developed many approaches to tractography aiming to reduce these problems and to represent the human physiology accurately.

  \section{Objectives}
  These approaches basically vary between local and global, deterministic and probabilistic and which approximations are made.

  For example, a form of local and deterministic tractography addressed in a previous work consisted on the application of the Runge-Kutta numerical integration until the diffusivity was very low or the edge of the dataset was reached, using trilinear interpolation.

  The example above is incapable to deal with the fiber crossing, it only follows one of the fibers in the best case. Furthermore, the image noises were just minimized by the interpolation.

  Trying to deal with these problems, the final objective is to find an global approach that takes into account probabilistic data. In this case, to develop a front evolution tractography taking into account the uncertainty probability distribution for each voxel as a threshold to determine if the fiber propagation should follow a path.

  Previous works on that line where developed \footnote{\ref{ch: biblio}: \textit{Campbell et al., 2005} and \textit{Decoteaux \& Deriche. 2008}} leading to positive results, but using a simpler probabilistic method, the QBall\footnote{\ref{ch: biblio}: Joahnsen-Berg \& Behrens, 2009}, which is based only on the direction probability for each voxel without information about the uncertainty for each measurement.

  \section{Challenges}
  The first big challenge for this tractography approach resides in its theory, which involves the Hamilton-Jacobi equation. This consists of partial differential equations and a classical mechanics formulation equivalent to the Newton's one, but with new concepts and different properties.

  Following, this technique demands resonance images more complex than the usual diffusion tensor (DTI). To obtain the uncertainty probability distribution for each voxel, a lot more information is needed. So, it uses diffusion spectrum images\footnote{\ref{ch: biblio}: Hagmann et al., 2006} (DSI).

  As this is one theme with high visual appeal, certainly it'll be necessary to dedicate a considerable effort to learn graphic libraries such as the VTK.

  Lastly, all this will demand lots of processing and memory. So, to make sure of the practical applicability of the technique to be developed, it'll be a challenge to parallelize as much as possible as the steps of the algorithm and, also minimize the consumed memory.

\chapter{Concepts and technologies to study}
  \section{Partial differential equations}
  Partial differential equations (PDEs) are equation of unknown functions in $\mathbb{R}^{n}$, $n \geq 1$, and it's partial derivatives. Wherem in the case $n = 1$, they the already known ordinary differential equations (ODEs).

  These equations are useful to model thermodynamics, fluids mechanic, waves and lots more.

    \subsection{Hamilton-Jacobi Equation}
    The Hamilton-Jacobi equation consists on an PDE, as described below (\ref{eq:hamilton-jacobi}).

    It's an classical mechanics formulation equivalent to Newton's one, useful on finding conserved movement quantities on a mechanics system. It's the only formulation for the mechanics where a particle can be described as a wave.

    It's defined as:

    \begin{equation} \label{eq:hamilton-jacobi}
      H + \frac{\partial S}{\partial t} = 0
    \end{equation}
    
    With:
    
    \begin{itemize}
      \item $H = H(s_{1}, ..., s_{n}, \frac{\partial S}{\partial s_{1}}, ..., \frac{\partial S}{\partial s_{n}})$
      \item $S = S(s_{1}, ..., s_{n}, t)$
    \end{itemize}
    
    Where, $s_{i}$, $1 \leq i \leq n$, are the $n$ generalized coordinates in the space and $t$ the time.

  \section{Diffusion spectrum images}
  Known as DSI, it's the most complex technique for magnet resonance diffusion imaging that provides the most information and details.

  By example, an usual diffusion tensor image (DTI) needs at least 6 different gradients to be obtained, while, in contrast, a DSI needs at least 50 independent gradient sets.

  Yet in this contrast between DTI and DSI, while the first one for each image voxel gives an three-dimensional tensor, the DSI, for each voxel, gives an uncertainty probability distribution for each direction, thus giving much more details and precision for the tractography.

  \section{Probabilistic tractography}
  This form of tractography is based on two probability distributions. One called dODF (diffusion Orientation Distribution Function) is the previous mentioned probability distribution of one fiber to follow a direction.

  The second one, uODF (uncertainty Orientation Distribution Function), quantifies the uncertainty of dODF. Using both together, it's possible to define thresholds for the fiber propagation, better dealing with noises and the fiber crossings.

  \section{Graphics libraries}
  As the final product are images, instead of developing all the structure for complex images reading and for displaying three-dimensional over the images, like cylinders representing the body fibers, is coherent to take advantage of an free software that already provides all these features.

  In the case, this software would be the MedSquare\footnote{\url{http://ccsl.ime.usp.br/medsquare/}}. Therefore, there will be an learning curve for it's architecture and the graphics libraries that are used on various tasks.

  Like GDCMfootnote{\url{http://gdcm.sourceforge.net/}} for DICOM reading or the VTK\footnote{\url{http://www.vtk.org/}} for three dimensional graphics generation. And others like Qt\footnote{\url{http://qt.digia.com/}} and ITK\footnote{\url{http://www.itk.org/}}.

  \section{Front evolution}
  The front evolution is an approach to tractography that searches the path of most diffusivity connecting two points in the image. With this objective, it always consider the global diffusivity on detriment of the local diffusivity. This is useful on minimizing the image noises, but has the downside of requiring an destination point.

  In general lines, the algorithm consists on successively expanding, through a cost function to reach each point, a set of points, called front, until there is no more points on the image outside of this set.

  In the beginning there is just the initial point on the front. Then, are calculated all the cost function values for every point in the front neighborhood. The point with the minimal cost will be included in the front for the next iteration.

  \section{Performance optimizations}
  Are expected large amounts of data and repetitive processing over this data for, therefore, the pretended approach to tractography. Thus, without doubts, there will come the moment where, in order to have something feasible for practical use, optimizations will be necessary.

  A way has been already proved it's efficiency is the GPGPU (General Programming on the Graphics processing Unit), which provides an highly parallelized environment and techniques to deal with large amounts of data. However, this depends on the problem adequacy to this environment, which will be one of the study objects.

  This dependency to environment adequacy is due to various factors among which the main ones are the data transference to the graphics processor through the PCI bus, besides it's processing cores architecture that were designed to repetitive tasks. It is, conditional branches compromises it's performance.

\chapter{Expected activities}
Basically, all the described activities in the following sections consist on MedSquare implementations, which, as mentioned before, will be the platform used as basis to the implementations due to it's structure that it already provides.

  \section{DSI reading}
  The MedSuquare already is capable of reading, among others, resonance images. However, it will be necessary to adapt it's reading structure to recognize a DSI and load.

  \section{Conventional probabilistic tractography implementation}
  With an loaded DSI, a good way to understand the probabilistic tractography background will be implementing it. Also, the MedSquare will get an robust tractography technique.

  \section{Conventional front evolution implementation}
  Again, to understand the method background, to implement it is a good path besides of again add a new feature to the MedSquare.

  \section{Front evolution cost function modifications}
  Understood both the probabilistic tractography and the front evolution, following, will be the moment to take the next step to what should be the differential of this study, joining both approaches.

  One way to do that, can be modifying the front evolution cost function so it takes in consideration the probability distributions. The path to that should be to adjust the cost function to give high costs for high uncertainty paths.

  \section{Practical comparisons between the three techniques}
  As a consequence, an study of the obtained results for each method with real images, will make possible to assess if the assumption that the union of the two techniques will bring gain on reliability to the tractography.

  \section{Optimization of the modified front evolution}
  Lastly, to make possible the practical usage of the developed methodology, will be important to optimize it's implementation, if possible with GPGPU.

\chapter{Expected results and products}
  \section{Detailed description of the approach to tractography}
  This description will be useful to anyone who wants to reproduce the obtained results and to expand it.

  \section{Comparative study about the obtained results with the new approach and the conventional ones}
  Be positive or negative, besides we are expecting positive results, the comparative results will be fundamental to, in addition to determine the methodology efficacy on dealing with noises and fiber crossings, to define future steps.

  \section{Tractography on the MedSquare}
  The expectation is that all the implementations above will be incorporated by this free software, thus consisting in one contribution to the community with a new feature.

\chapter{Conclusion}
It's expected that the union of the probabilistic tractography with the front evolution to provide more reliable results, in respect to noises and fiber crossings, than the usual methodologies.

Thus, justifying all the work on theoretical studies and software implementations, as the meanings to have a mean to compare which gains or losses the proposed methodology will bring.

\chapter{References} \label{ch: biblio}

\begin{itemize}
  \item JOAHNSEN-BERG, Heidi; BEHRENS, Timothy E. J. \textit{Diffusion MRI}. 1st edition. May 2009;
  \item JACKOWSKI, Marcel; KAOD, Chiu Yen; QIUA, Maolin; CONSTABLE, R. Todd; STAIB, Lawrence H. \textit{White matter tractography by anisotropic wavefront evolution and diffusion tensor imaging}. Med Image Anal. 2005 October;
  \item Hagmann P, Jonasson L, Maeder P, Thiran JP, van Wedeen J, Meuli R. Understanding diffusion MR imaging techniques: from scalar diffusion-weighted imaging to diffusion tensor imaging and beyond. RadioGraphics 2006;26(suppl 1):S205–S223.
  \item DESCOTEAUX, Maxime; DERICHE, Rachid. \textit{High Angular Resolution Diffusion MRI Segmentation Using
Region-Based Statistical Surface Evolution}. J Math Imaging Vis. 2008;
  \item CAMPBELL, Jennifer S. W.; SIDDIQI, Kaleem; RYMAR, Vladmir V.; SADIKOT, Abbas F.; PIKE, G. Bruce. \textit{Flow-based fiber tracking with diffusion tensor and q-ball data: Validation and comparison to principal diffusion direction techniques}. NeuroImage 27 (2005) 725 – 736;
  \item Parker, G., Wheeler-Kingshott, C., Barker, G., 2002a. Estimating distributed anatomical connectivity using fast marching methods and diffusion tensor imaging. IEEE Trans. Med. Imaging 21 (5), 505 – 512.
\end{itemize}

\end{document}