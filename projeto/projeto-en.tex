\documentclass[a4paper,11pt]{report}
\usepackage[T1]{fontenc}
\usepackage[utf8]{inputenc}
\usepackage{lmodern}
\usepackage[english]{babel}
\usepackage{amsfonts}
\usepackage{hyperref}

\title{MSc project\\
       \textbf{Front evolution tractography making use of uncertainty probability distributions}}
\author{Rafael Reggiani Manzo\\
        \textbf{Supervisor:} Prof. Doutor Marcel Parolin Jackowski}

\begin{document}

\maketitle
\tableofcontents

\begin{abstract}
Between the many techniques used for medical image exploration, tractography is based on magnetic resonance images in order to reconstruct in a three-dimensional model the fibers of the human body. It is the only non-invasive technique available for human brain studies.

Two of these algorithms are the probabilist tractography and the front evolution tractography, successfully satisfying the objective above. So, is raised the question about how to incorporate the uncertainty probability distributions from the first method on the second one and which kind of results would be possible to achieve.

To answer these questions, an whole theoretical study about the methods must be done, because these methods advanced physics and calculus topics. After studying these methods, will come a stage of implementation, where is expected to be able to, in fact, assimilate both the methodologies.

To finalize, after dominate both techniques, it'll be possible to make the described method combination and to study it's results compared to the other methods.
\end{abstract}

\chapter{Introduction}
  \section{Motivation}
  Tractography is a medical image exploration computational technique that aims, through many different approaches reconstruct the brain white matter fibers (axonal fibers involved on myelin). Being the only non invasive technique for observation \textit{in vivo} of the human brain and one of the most promissory tools to aid in it's study.

  This technique uses as data source magnetic resonance images. In virtue of the white matter nature, where the water molecules diffusivity is more intense in the fiber orientation. This property is called anisotropy.

  On the other hand, the magnetic resonance is subject to noise and it doesn't represents with precision the fiber crossing. So, over time, were developed many approaches to tractography aiming to reduce these problems and to represent the human physiology accurately.

  \section{Objectives}
  These approaches basically vary between local and global, deterministic and probabilistic and which approximations are made.

  For example, a form of local and deterministic tractography addressed in a previous work consisted on the application of the Runge-Kutta numerical integration until the diffusivity was very low or the edge of the dataset was reached, using trilinear interpolation.

  The example above is incapable to deal with the fiber crossing, it only follows one of the fibers in the best case. Furthermore, the image noises were just minimized by the interpolation.

  Trying to deal with these problems, the final objective is to find an global approach that takes into account probabilistic data. In this case, to develop a front evolution tractography taking into account the uncertainty probability distribution for each voxel as a threshold to determine if the fiber propagation should follow a path.

  Previous works on that line where developed \footnote{\ref{ch: biblio}: \textit{Campbell et al., 2005} and \textit{Decoteaux \& Deriche. 2008}} leading to positive results, but using a simpler probabilistic method, the QBall\footnote{\ref{ch: biblio}: Joahnsen-Berg \& Behrens, 2009}, which is based only on the direction probability for each voxel without information about the uncertainty for each measurement.

  \section{Challenges}
  The first big challenge for this tractography approach resides in its theory, which involves the Hamilton-Jacobi equation. This consists of partial differential equations and a classical mechanics formulation equivalent to the Newton's one, but with new concepts and different properties.

  Following, this technique demands resonance images more complex than the usual diffusion tensor (DTI). To obtain the uncertainty probability distribution for each voxel, a lot more information is needed. So, it uses diffusion spectrum images\footnote{\ref{ch: biblio}: Hagmann et al., 2006} (DSI).

  As this is one theme with high visual appeal, certainly it'll be necessary to dedicate a considerable effort to learn graphic libraries such as the VTK.

  Lastly, all this will demand lots of processing and memory. So, to make sure of the practical applicability of the technique to be developed, it'll be a challenge to parallelize as much as possible as the steps of the algorithm and, also minimize the consumed memory.
  
\chapter{References} \label{ch: biblio}

\begin{itemize}
  \item JOAHNSEN-BERG, Heidi; BEHRENS, Timothy E. J. \textit{Diffusion MRI}. 1st edition. May 2009;
  \item JACKOWSKI, Marcel; KAOD, Chiu Yen; QIUA, Maolin; CONSTABLE, R. Todd; STAIB, Lawrence H. \textit{White matter tractography by anisotropic wavefront evolution and diffusion tensor imaging}. Med Image Anal. 2005 October;
  \item Hagmann P, Jonasson L, Maeder P, Thiran JP, van Wedeen J, Meuli R. Understanding diffusion MR imaging techniques: from scalar diffusion-weighted imaging to diffusion tensor imaging and beyond. RadioGraphics 2006;26(suppl 1):S205–S223.
  \item DESCOTEAUX, Maxime; DERICHE, Rachid. \textit{High Angular Resolution Diffusion MRI Segmentation Using
Region-Based Statistical Surface Evolution}. J Math Imaging Vis. 2008;
  \item CAMPBELL, Jennifer S. W.; SIDDIQI, Kaleem; RYMAR, Vladmir V.; SADIKOT, Abbas F.; PIKE, G. Bruce. \textit{Flow-based fiber tracking with diffusion tensor and q-ball data: Validation and comparison to principal diffusion direction techniques}. NeuroImage 27 (2005) 725 – 736;
  \item Parker, G., Wheeler-Kingshott, C., Barker, G., 2002a. Estimating distributed anatomical connectivity using fast marching methods and diffusion tensor imaging. IEEE Trans. Med. Imaging 21 (5), 505 – 512.
\end{itemize}

\end{document}