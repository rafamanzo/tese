\documentclass[a4paper,11pt]{report}
\usepackage[T1]{fontenc}
\usepackage[utf8]{inputenc}
\usepackage{lmodern}
\usepackage[english]{babel}
\usepackage{amsfonts}
\usepackage{hyperref}

\title{MSc project\\
       \textbf{Front evolution tractography making use of uncertainty probability distributions}}
\author{Rafael Reggiani Manzo\\
        \textbf{Supervisor:} Prof. Doutor Marcel Parolin Jackowski}

\begin{document}

\maketitle
\tableofcontents

\begin{abstract}
Between the many techniques used for medical image exploration, tractography is based on magnetic resonance images in order to reconstruct in a three-dimensional model the fibers of the human body. It is the only non-invasive technique available for human brain studies.

Two of these algorithms are the probabilist tractography and the front evolution tractography, successfully satisfying the objective above. So, is raised the question about how to incorporate the uncertainty probability distributions from the first method on the second one and which kind of results would be possible to achieve.

To answer these questions, an whole theoretical study about the methods must be done, because these methods advanced physics and calculus topics. After studying these methods, will come a stage of implementation, where is expected to be able to, in fact, assimilate both the methodologies.

To finalize, after dominate both techniques, it'll be possible to make the described method combination and to study it's results compared to the other methods.
\end{abstract}

\chapter{Introduction}

\chapter{References} \label{ch: biblio}

\begin{itemize}
  \item JOAHNSEN-BERG, Heidi; BEHRENS, Timothy E. J. \textit{Diffusion MRI}. 1st edition. May 2009;
  \item JACKOWSKI, Marcel; KAOD, Chiu Yen; QIUA, Maolin; CONSTABLE, R. Todd; STAIB, Lawrence H. \textit{White matter tractography by anisotropic wavefront evolution and diffusion tensor imaging}. Med Image Anal. 2005 October;
  \item Hagmann P, Jonasson L, Maeder P, Thiran JP, van Wedeen J, Meuli R. Understanding diffusion MR imaging techniques: from scalar diffusion-weighted imaging to diffusion tensor imaging and beyond. RadioGraphics 2006;26(suppl 1):S205–S223.
  \item DESCOTEAUX, Maxime; DERICHE, Rachid. \textit{High Angular Resolution Diffusion MRI Segmentation Using
Region-Based Statistical Surface Evolution}. J Math Imaging Vis. 2008;
  \item CAMPBELL, Jennifer S. W.; SIDDIQI, Kaleem; RYMAR, Vladmir V.; SADIKOT, Abbas F.; PIKE, G. Bruce. \textit{Flow-based fiber tracking with diffusion tensor and q-ball data: Validation and comparison to principal diffusion direction techniques}. NeuroImage 27 (2005) 725 – 736;
  \item Parker, G., Wheeler-Kingshott, C., Barker, G., 2002a. Estimating distributed anatomical connectivity using fast marching methods and diffusion tensor imaging. IEEE Trans. Med. Imaging 21 (5), 505 – 512.
\end{itemize}

\end{document}