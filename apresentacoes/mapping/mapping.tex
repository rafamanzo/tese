\documentclass[10pt]{beamer}
\usepackage[T1]{fontenc}
\usepackage[utf8]{inputenc}
\usepackage{lmodern}
\usepackage[english]{babel}
\usepackage{hyperref}
\usepackage{listings}
\usetheme{JuanLesPins}

\makeatletter
\setbeamertemplate{footline}
{
  \leavevmode%
  \hbox{%
  \begin{beamercolorbox}[wd=\paperwidth,ht=2.25ex,dp=1ex,right]{date in head/foot}%
    \insertframenumber\hspace*{2ex} 
  \end{beamercolorbox}}%
  \vskip0pt%
}
\makeatother

\title{Classifying DW-MRI voxels into isotropic, Gaussian anisotropic or non-Gaussian anisotropic for then enable fibre-tracking to overcome the former one}
\author{Rafael Reggiani Manzo}

\begin{document}
\maketitle

\section{Introduction}

\begin{frame}
  \frametitle{}
  \framesubtitle{}

  \begin{Large}
  \begin{center}
  \textbf{Introduction}
  \end{center}
  \end{Large}
\end{frame}

\begin{frame}
  \frametitle{Motivation}
  \framesubtitle{Tractography}

  \begin{itemize}
    \item Tractography (also known as fibre-tracking) is one of the unique techniques that permits \textit{in\-vivo} study of the human brain anatomy and other body structures like muscles by virtually reconstructing their fibres
    \item It is based on diffusion magnetic resonance images information
      \begin{itemize}
        \item The diffusion process in each voxel is described by a tensor, which assumes that this process follow a Gaussian distribution
        \item This assumption is not true when there are crossings, splits, kissing or even high curvature fibres
      \end{itemize}
    \item It has several precision issues and limitations related to this imprecise representation
  \end{itemize}
\end{frame}

\begin{frame}
  \frametitle{Objective}
  \framesubtitle{}

  \begin{itemize}
    \item 
  \end{itemize}
\end{frame}

\begin{frame}
  \frametitle{Articles}
  \framesubtitle{}

  \begin{itemize}
    \item ``ALEXANDER, D. C.; BARKER, G. J.; ARRIDGE, S. R. Detection and modeling of non-Gaussian apparent diffusion coefficient profiles in human brain data. Magn Reson Med, v. 48, n. 2, p. 331-40, Aug 2002.''
  \end{itemize}
\end{frame}

\end{document}