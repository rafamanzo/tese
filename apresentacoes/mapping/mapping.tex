\documentclass[10pt]{beamer}
\usepackage[T1]{fontenc}
\usepackage[utf8]{inputenc}
\usepackage{lmodern}
\usepackage[english]{babel}
\usepackage{hyperref}
\usepackage{listings}
\usetheme{JuanLesPins}

\makeatletter
\setbeamertemplate{footline}
{
  \leavevmode%
  \hbox{%
  \begin{beamercolorbox}[wd=\paperwidth,ht=2.25ex,dp=1ex,right]{date in head/foot}%
    \insertframenumber\hspace*{2ex} 
  \end{beamercolorbox}}%
  \vskip0pt%
}
\makeatother

\title{Classifying DW-MRI voxels into isotropic, Gaussian anisotropic or non-Gaussian anisotropic for then enable fibre-tracking to overcome the former one}
\author{Rafael Reggiani Manzo}

\begin{document}
\maketitle

\section{Introduction}

\begin{frame}
  \frametitle{}
  \framesubtitle{}

  \begin{Large}
  \begin{center}
  \textbf{Introduction}
  \end{center}
  \end{Large}
\end{frame}

\begin{frame}
  \frametitle{Motivation}
  \framesubtitle{Tractography}

  \begin{itemize}
    \item Tractography (also known as fibre-tracking) is one of the unique techniques that permits \textit{in-vivo} study of the human brain anatomy and other body structures like muscles by virtually reconstructing their fibres
    \item It is based on diffusion magnetic resonance images information
      \begin{itemize}
        \item The diffusion process in each voxel is described by a tensor, which assumes that this process follow a Gaussian distribution
        \item This assumption is not true when there are crossings, splits, kissing or even high curvature fibres
      \end{itemize}
    \item It has several precision issues and limitations related to this imprecise representation
  \end{itemize}
\end{frame}

\begin{frame}
  \frametitle{Objective}
  \framesubtitle{}

  \begin{enumerate}
    \item Given a previously calculated map of those voxels on which the Gaussian model does not fit
    \item Overcome them by looking at their neighbourhood (global methods)
  \end{enumerate}
\end{frame}

\begin{frame}
  \frametitle{Articles}
  \framesubtitle{}

  \begin{enumerate}
    \item ``ALEXANDER, D. C.; BARKER, G. J.; ARRIDGE, S. R. Detection and modeling of non-Gaussian apparent diffusion coefficient profiles in human brain data. Magn Reson Med, v. 48, n. 2, p. 331-40, Aug 2002.''
    \item ``PARKER, G. J.; ALEXANDER, D. C. Probabilistic Monte Carlo based mapping of cerebral connections utilising whole-brain crossing fibre information. Inf Process Med Imaging, v. 18, p. 684-95, Jul 2003.''
  \end{enumerate}
\end{frame}

\section{Detection and modeling of non-Gaussian apparent diffusion coefficient profiles in human brain data}
\begin{frame}
  \frametitle{}
  \framesubtitle{}

  \begin{Large}
  \begin{center}
  \textbf{Detection and modeling of non-Gaussian apparent diffusion coefficient profiles in human brain data}
  \end{center}
  \end{Large}
\end{frame}

\begin{frame}
  \frametitle{Base}
  \framesubtitle{}

  \begin{itemize}
    \item Consider that on each voxel the diffusion may be described by a probability density function for every possible direction
    \item So it can be understood as a spheric function
    \item Spheric functions can be described by the Spheric Harmonic Series
  \end{itemize}
\end{frame}

\begin{frame}
  \frametitle{Voxel classification procedure}
  \framesubtitle{}

  \begin{enumerate}
    \item Start truncating the series at order $n = 0$
    \item Using a hypothesis test check if the order $n$ is as descriptive than the order $n + 2$
      \begin{itemize}
        \item If the null hypothesis is rejected, repeat the procedure for $n = n + 2$
        \item Otherwise, stop and classify the voxel accordingly
      \end{itemize}
  \end{enumerate}

  If the series was truncated at\footnote{Odd orders of the series are zero for the tensor image case}:
  \begin{itemize}
    \item \textbf{Order 0}: isotropy
    \item \textbf{Order 2}: Gaussian anisotropy
    \item \textbf{Order 4} or higher: non-Gaussian anisotropy
  \end{itemize}
\end{frame}

\begin{frame}

  \frametitle{Results}
  \framesubtitle{}

  In order to validate this methodology were made tests:

  \begin{itemize}
    \item \textbf{Synthetic data}
      \begin{itemize}
        \item 92\% of the Gaussian anisotropy voxels were correctly classified (other 8\% were overfitted to non-Gaussian)
        \item 97\% of the non-Gaussian anisotropy voxels (fibre crossing) were correctly classified (other 3\% were underfitted to Gaussian)
      \end{itemize}
    \item \textbf{Human brain data}
      \begin{itemize}
        \item Was not made a detailed anatomical analysis
        \item But it looks like it correctly separate isotropic from anisotropic regions
        \item Highlights regions like the \textit{corona radiata} and the crossing between optic radiation fibres and \textit{corpus callosum} ones are correctly classified
      \end{itemize}
  \end{itemize}
\end{frame}

\section{Probabilistic Monte Carlo based mapping of cerebral connections utilising whole-brain crossing fibre information}
\begin{frame}
  \frametitle{}
  \framesubtitle{}

  \begin{Large}
  \begin{center}
  \textbf{Probabilistic Monte Carlo based mapping of cerebral connections utilising whole-brain crossing fibre information}
  \end{center}
  \end{Large}
\end{frame}

\begin{frame}
  \frametitle{Base}
  \framesubtitle{}

  \begin{itemize}
    \item Applies the method of the first article to find non-Gaussian anisotropy
    \item Then fits a multi-Gaussian model with $n=2$ on these voxels
    \item Based on this new model, produce a new Probability Density Function (PDF) to be used for discrete and probabilistic tractography
  \end{itemize}
\end{frame}

  \subsection{Discussion}
  \begin{frame}
    \frametitle{Limitations}
    \framesubtitle{}

    \begin{itemize}
      \item Validation of the method is to date limited to cross-referencing the results
with know anatomy
        \begin{itemize}
          \item Knowledge of anatomy may be incomplete, and quantification of errors is difficult
          \item Some connections produced are truly known and validated, but others are unexpected and not validated
          \item It is not possible to distinguish between those two kinds of connections
        \end{itemize}
    \end{itemize}
  \end{frame}

  \begin{frame}
    \frametitle{Next steps}
    \framesubtitle{}

    This would be improved by:
    \begin{itemize}
      \item Reduced data noise
      \item Introduction of constraints on tracking process
    \end{itemize}

    Citations to this article consist of:
    \begin{itemize}
      \item Comparison with a different technique for DTI
        \begin{itemize}
          \item ``T.E.J. Behrens; H. Johansen Berg; S. Jbabdi; M.F.S. Rushworth; and M.W. Woolrich. Probabilistic diffusion tractography with multiple fibre orientations: What can we gain? NeuroImage 34 (2007) 144 – 155''
        \end{itemize}
      \item Comparison with a HARDI technique
        \begin{itemize}
          \item ``M. Descoteaux; R. Deriche; T. Knösche; and A. Anwander. Deterministic and Probabilistic Tractography based on Complex Fibre Orientation Distributions. Medical Imaging, IEEE Transactions on  (Volume:28 ,  Issue: 2 ). 2009''
        \end{itemize}
    \end{itemize}
  \end{frame}

\end{document}