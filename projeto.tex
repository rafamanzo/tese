\documentclass[a4paper,11pt]{report}
\usepackage[T1]{fontenc}
\usepackage[utf8]{inputenc}
\usepackage{lmodern}
\usepackage[brazil]{babel}

\title{Projeto de mestrado\\
       \textbf{Tractografia por evolução de frentes levando em conta distribuições de probabilidade de incerteza}}
\author{Rafael Reggiani Manzo\\
        \textbf{Orientador:} Prof. Doutor Marcel Parolin Jackowski}

\begin{document}

\maketitle
\tableofcontents

\begin{abstract}
\end{abstract}

\chapter{Introdução}
  \section{Motivações}
  A tractografia é uma técnica computacional que busca através de várias abordagens diferentes reconstruir as fibras da substância branca do cérebro (fibras axonais envoltas em mielina). Sendo a única técnica não invasiva para estudo \textit{in vivo} do cérebro humano e uma das ferramentas mais promissoras para auxiliar em seu estudo.
  
  Esta técnica utiliza como fonte de dados imagens de ressonância magnética. Em virtude da natureza da substância branca onde a difusão das moléculas de água é muito mais intensa no sentido de suas fibras. Propriedade chamada de anisotropia.
  
  Por outro lado, a ressonância magnética é sujeita a ruídos e não representa com precisão o cruzamento de duas fibras. Dessa forma, foram desenvolvidas ao longo do tempo diversas abordagens à tractografia buscando minimizar estes problemas para representar com a maior fidelidade possível a fisiologia humana.
  
  \section{Objetivos}
  Estas abordagens basicamente variam em locais ou globais, determinísticas ou probabilísticas e quais aproximações são feitas. 
  
  Por exemplo, uma forma forma de tractografia local e determinística abordada em um trabalho anterior, consistia em aplicar a técnica de integração numérica de Runge-Kutta até que a difusividade fosse muito baixa ou uma fronteira do conjunto de dados fosse alcançada, utilizando interpolação trilinear para aproximar a difusividade entre voxels.
  
  O exemplo acima então é incapaz de lidar com o cruzamento de fibras os ruídos da imagem eram apenas minimizados com a interpolação feita.
  
  Tentando lidar melhor com estes problemas, o objetivo final é chegar a uma abordagem global que leve em conta dados probabilísticos. No caso, desenvolver a tractografia por evolução de frentes levando em conta a distribuição de probabilidade da incerteza em cada voxel como um limiar para identificar se a propagação de uma fibra deve seguir determinado caminho.
  
  \section{Desafios}
  O primeiro grande desafio desta abordagem à tractografia está na sua teoria, que envolve a equação de Hamilton-Jacobi. Portanto, equações diferenciais parciais e uma formulação de mecânica clássica equivalente à de Newton, mas com conceitos novos e propriedades diferentes.
  
  Em seguida, esta técnica exige imagens de ressonância mais complexas que as usuais por tensores de difusão (\textit{DTI}). Para obter a distribuição de probabilidade da incerteza em cada voxel, muito mais informação é necessária. Então, são utilizadas imagens de ressonância, imagens por espectro de difusão (DSI).
  
  Por fim, tudo isto exigirá muito processamento e memória. Então, para garantir a aplicabilidade da técnica a ser desenvolvida na prática, será um desafio paralelizar ao máximo todas as etapas do algoritmo e ainda minimizar o consumo de memória.
  
\chapter{Conceitos e tecnologias a estudar}

\chapter{Atividades esperadas}

\chapter{Resultados e produtos esperados}

\chapter{Conclusão}

\chapter{Referências Bibliográficas}

\begin{itemize}
  \item JOAHNSEN-BERG, Heidi; BEHRENS, Timothy E. J. \textit{Diffusion MRI};
  \item JACKOWSKI, Marcel; KAOD, Chiu Yen; QIUA, Maolin; CONSTABLE, R. Todd; STAIB, Lawrence H. \textit{White matter tractography by anisotropic wavefront evolution and diffusion tensor imaging}. Med Image Anal. 2005 October;
\end{itemize}

\end{document}
