\documentclass[a4paper,11pt]{report}
\usepackage[T1]{fontenc}
\usepackage[utf8]{inputenc}
\usepackage{lmodern}
\usepackage[brazil]{babel}
\usepackage{amsfonts}
\usepackage{hyperref}

\title{Projeto de mestrado\\
       \textbf{Tractografia por evolução de frentes levando em conta distribuições de probabilidade de incerteza}}
\author{Rafael Reggiani Manzo\\
        \textbf{Orientador:} Prof. Doutor Marcel Parolin Jackowski}

\begin{document}

\maketitle
\tableofcontents

\begin{abstract}
\end{abstract}

\chapter{Introdução}
  \section{Motivações}
  A tractografia é uma técnica computacional que busca através de várias abordagens diferentes reconstruir as fibras da substância branca do cérebro (fibras axonais envoltas em mielina). Sendo a única técnica não invasiva para observação \textit{in vivo} do cérebro humano e uma das ferramentas mais promissoras para auxiliar em seu estudo.
  
  Esta técnica utiliza como fonte de dados imagens de ressonância magnética. Em virtude da natureza da substância branca onde a difusão das moléculas de água é muito mais intensa no sentido de suas fibras. Propriedade chamada de anisotropia.
  
  Por outro lado, a ressonância magnética é sujeita a ruídos e não representa com precisão o cruzamento de duas fibras. Dessa forma, foram desenvolvidas ao longo do tempo diversas abordagens à tractografia buscando minimizar estes problemas para representar com a maior fidelidade possível a fisiologia humana.
  
  \section{Objetivos}
  Estas abordagens basicamente variam em locais ou globais, determinísticas ou probabilísticas e quais aproximações são feitas. 
  
  Por exemplo, uma forma forma de tractografia local e determinística abordada em um trabalho anterior, consistia em aplicar a técnica de integração numérica de Runge-Kutta até que a difusividade fosse muito baixa ou uma fronteira do conjunto de dados fosse alcançada, utilizando interpolação trilinear para aproximar a difusividade entre voxels.
  
  O exemplo acima então é incapaz de lidar com o cruzamento de fibras os ruídos da imagem eram apenas minimizados com a interpolação feita.
  
  Tentando lidar melhor com estes problemas, o objetivo final é chegar a uma abordagem global que leve em conta dados probabilísticos. No caso, desenvolver a tractografia por evolução de frentes levando em conta a distribuição de probabilidade da incerteza em cada voxel como um limiar para identificar se a propagação de uma fibra deve seguir determinado caminho.
  
  \section{Desafios}
  O primeiro grande desafio desta abordagem à tractografia está na sua teoria, que envolve a equação de Hamilton-Jacobi. Portanto, equações diferenciais parciais e uma formulação de mecânica clássica equivalente à de Newton, mas com conceitos novos e propriedades diferentes.
  
  Em seguida, esta técnica exige imagens de ressonância mais complexas que as usuais por tensores de difusão (\textit{DTI}). Para obter a distribuição de probabilidade da incerteza em cada voxel, muito mais informação é necessária. Então, são utilizadas imagens de ressonância, imagens por espectro de difusão (DSI).
  
  Como este é um tema com alto apelo visual, certamente será preciso dedicar um esforço considerável para dominar bibliotecas gráficas como a \textit{VTK}. O que já se provou um desafio no trabalho anterior já mencionado.
  
  Por fim, tudo isto exigirá muito processamento e memória. Então, para garantir a aplicabilidade da técnica a ser desenvolvida na prática, será um desafio paralelizar ao máximo todas as etapas do algoritmo e ainda minimizar o consumo de memória.
  
\chapter{Conceitos e tecnologias a estudar}
  \section{Equações diferenciais parciais}
    Equações diferenciais parciais (EDPs) são equações de funções desconhecidas no $\mathbb{R}^{n}$, $n \geq 1$, e suas derivadas parciais. Onde o caso no $n = 1$, são as já conhecidas equações diferenciais ordinárias (EDOs).
    
    Estas equações são úteis para modelar termodinâmica, mecânica dos fluídos, ondas e muitas outras aplicações.
  
    \subsection{Equação de Hamilton-Jacobi}
    A equação de Hamilton-Jacobi (\ref{eq:hamilton-jacobi}) é uma EDP, conforme descrita abaixo. É uma formulação da mecânica clássica equivalente à de Newton, útil para encontrar quantidades de movimento conservadas em um sistema mecânico. É a única formulação da mecânica onde uma partícula pode ser representada como uma onda.
    
    Ela é definida como: 

    \begin{equation} \label{eq:hamilton-jacobi}
      H + \frac{\partial S}{\partial t} = 0
    \end{equation}
    
    Com:
    
    \begin{itemize}
      \item $H = H(s_{1}, ..., s_{n}, \frac{\partial S}{\partial s_{1}}, ..., \frac{\partial S}{\partial s_{n}})$
      \item $S = S(s_{1}, ..., s_{n}, t)$
    \end{itemize}
    
    Onde, $s_{i}$, $1 \leq i \leq n$, são as $n$ coordenadas generalizadas no espaço e $t$ o tempo.
    
    
  \section{Imagens por espectro de difusão}
  Conhecida como DSI (\textit{Diffusion Spectrum Imaging}), é a mais complexa técnica de imagem de difusão por ressonância magnética que provê a maior quantidade de informação e detalhes.

  Por exemplo, uma imagem por tensores de difusão (DTI) mais usual precisa de no mínimo 6 gradientes diferentes para ser obtida, enquanto que em contraste uma DSI precisa de no mínimo 50 conjuntos diferentes de gradientes.

  Ainda neste contraste entre DTI e DSI, enquanto a primeira para cada voxel da imagem fornece um tensor tridimensional, a DSI para cada voxel tem uma distribuição de probabilidade sobre a incerteza de cada direção neste, assim permitindo muito mais detalhe e precisão na tractografia.

  \section{Tractografia probabilística}
  Esta forma de tractografia se baseia em duas distribuições de probabilidade. Uma chamada de dODF (\textit{diffusion Orientation Distribution Function}) é a distribuição de probabilidade, já mencionada, de uma fibra seguir determinada direção em um voxel.
  
  A segunda uODF (\textit{uncertainty Orientation Distribuition Function}) quantifica a incerteza sobre a dODF. Utilizando ambas em conjunto é possível definir limiares para propagações de fibras, lidando melhor com ruídos e cruzamentos de fibras.
  
  \section{Bibliotecas gráficas}
  Como o produto final são imagens, ao invés de desenvolver toda a estrutura de leitura de imagens complexas e exibição de elementos tridimensionais sobre estas imagens, como cilindros representando as fibras do corpo, é coerente aproveitar um software livre já com suporte a todas estas funcionalidades.
  
  No caso, este software seria o MedSquare \footnote{\url{http://ccsl.ime.usp.br/medsquare/}}. Para tanto existirá uma curva de aprendizado de sua arquitetura e as bibliotecas gráficas que são utilizadas em diversas tarefas.
  
  Como GDCM\footnote{\url{http://gdcm.sourceforge.net/}} para leitura de DICOM ou VTK\footnote{\url{http://www.vtk.org/}} para geração de gráficos tridimensionais. Entre outras como Qt\footnote{\url{http://qt.digia.com/}} e ITK\footnote{\url{http://www.itk.org/}}.
  
  \section{Evolução de frentes}
  A evolução de frentes é uma abordagem à tractografia que busca o caminho de maior difusividade ligando dois pontos na imagem. Com este objetivo, ela sempre considera a difusividade global em detrimento da local. Isto é útil para minimizar ruídos da imagem, mas tem o lado negativo de necessitar de um ponto de destino.
  
  Em linhas gerais o algoritmo consiste em sucessivamente expandir, por meio de uma função de custos para atingir cada ponto, um conjunto de pontos, chamado de frente, até que não exista mais ponto algum na imagem fora deste conjunto.
  
  No início apenas o ponto inicial está na frente. Então são calculados os valores de função de custo para todos os pontos vizinhos à frente. O ponto de menor custo passará a fazer parte da frente na próxima iteração.
  
  \section{Otimizações de performance}
  São esperados grandes quantidades de dados e processamentos repetitivos sobre estes dados para, por fim, alcançar a abordagem à tractografia pretendida. Então, sem dúvida, irá chegar um momento onde para ter algo viável no uso prático em que otimizações serão necessárias.
  
  Uma forma que já se provou eficiente foi a programação para GPGPU (General Programming on the Graphics Processing Unit), que fornece um ambiente altamente paralelizado e técnicas para lidar com grandes quantidades de dados. Porém, isto depende da adequação do problema ao ambiente, que será um dos objetos de estudo.
  
  Esta dependência de adequação ao ambiente se deve a diversos fatores dentre os quais os principais são a transferência de dados para o processador gráfico que tem de passar pelo PCI além da arquitetura dos núcleos de processamento da placa gráfica que foram feitos para tarefas repetitivas. Ou seja, desvios condicionais, por exemplo, reduzem a performance do processamento.

\chapter{Atividades esperadas}
  \section{Leitura de DSI}
  
  \section{Implementação de tractografia probabilística convencional}
  
  \section{Implementação de evolução de frentes convencional}
  
  \section{Modificações na função de custos da evolução de frentes}
  
  \section{Comparações práticas entre as três técnicas}
  
  \section{Otimização da evolução de frentes modificada}

\chapter{Resultados e produtos esperados}
  \section{Descrição detalhada de uma nova abordagem à tractografia}
  
  \section{Estudo comparando os resultados obtidos com a abordagem desenvolvida às convencionais}
  
  \section{Tractografia no MedSquare}

\chapter{Conclusão}

\chapter{Referências Bibliográficas}

\begin{itemize}
  \item JOAHNSEN-BERG, Heidi; BEHRENS, Timothy E. J. \textit{Diffusion MRI};
  \item JACKOWSKI, Marcel; KAOD, Chiu Yen; QIUA, Maolin; CONSTABLE, R. Todd; STAIB, Lawrence H. \textit{White matter tractography by anisotropic wavefront evolution and diffusion tensor imaging}. Med Image Anal. 2005 October;
  \item Hagmann P, Jonasson L, Maeder P, Thiran JP, van Wedeen J, Meuli R. Understanding diffusion MR imaging techniques: from scalar diffusion-weighted imaging to diffusion tensor imaging and beyond. RadioGraphics 2006;26(suppl 1):S205–S223.
\end{itemize}

\end{document}
